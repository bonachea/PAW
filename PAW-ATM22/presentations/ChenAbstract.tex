\documentclass{article}

\title{The Convergence of Exascale Computing, Data Science and Visualization Towards Zero-carbon Fuels for Power 
and Transportation}
\author{Jacqueline H. Chen\thanks{Sandia National Laboratories}}
\date{November 14, 2022}
\providecommand{\keywords}[1]{\textbf{\textit{Keywords---}} #1}
\begin{document}

\maketitle

% Here is the abstract.
\begin{abstract}
    Mitigating climate change while providing the nation’s transportation and power generation are important to energy and environmental security.  The shift to hydrogen as a clean energy carrier is one of the most promising strategies to reduce CO2 emissions in the face of increasing energy demand.  While hydrogen has a few drawbacks as an energy carrier due to its low energy density, ammonia is simpler to transport and store for extended periods of time, making it an attractive carbon-free energy carrier for off-grid localized power generation and marine shipping.  However ammonia has poor reactivity and forms NOx and N2O emissions.  The poor ammonia reactivity can be circumvented by partial cracking of ammonia to form ammonia/hydrogen/nitrogen blends tailored to match conventional hydrocarbon fuel properties. However, combustion of ammonia/hydrogen/nitrogen blends at high pressure, and in particular, the coupling between turbulence and fast hydrogen diffusion remains poorly understood. Pre-exascale computing provides a unique opportunity for direct numerical simulation (DNS) of turbulent combustion with ammonia/hydrogen blends to investigate the pressure effects on combustion rate, blow-off limits and chemical pathways for NOx and N2O formation.

    Exascale computing introduces challenges for data management and the need for reduced order surrogate models (ROMS) for chemical species dimension reduction and for novel in situ analysis and visualization methods.  A novel model driven on-the-fly ROM recently formulated and implemented in reactive flow DNS to reduce the computational cost of chemistry will be described.    Recent advances in topological segmentation, feature extraction, and statistical summarization for extreme-scale data will be discussed in the context of in situ analysis workflows that capture salient time-varying features.
    
\end{abstract}

%\keywords{}
\end{document}