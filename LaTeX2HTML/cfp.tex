\documentclass[12pt]{article}

\usepackage{graphicx}  
\usepackage{amsmath}
\usepackage{setspace}
\usepackage{fancyvrb,relsize}
\usepackage{relsize}
\usepackage{verbatim}
\usepackage{pdfpages}
\usepackage{wrapfig}
\usepackage{fancyvrb,relsize}
\usepackage{subfigure}
\usepackage{color}
% \usepackage{lipsum}
\usepackage[showframe=false]{geometry}
\geometry{verbose,tmargin=90pt,bmargin=90pt,lmargin=90pt,
rmargin=90pt}

\title{Call for Papers \\
PGAS Applications Workshop (PAW) \\
Held in conjunction with SC 16, Salt Lake City, UT} 

\date{}

\begin{document}


\maketitle

% Call for Papers 
% PGAS Applications Workshop (PAW)
% Held in conjunction with SC 16, Salt Lake City, UT

\section{Summary}

The race towards Exascale computing is on, and a lot of stress is put
on researchers to break the boundaries of productivity and efficiency
imposed by traditional programming models. 
Partitioned Global Address Space (PGAS) languages are  an
effective alternative, and the most promising path towards 
sustainable programming environments for exascale machines. 
Languages such as UPC, Fortran, Chapel, and X10 are now more widely
available than ever, thanks to increased support from vendors and
open-source communities. PGAS models also take the form of meta-languages 
and libraries, such as Unified Parallel C++ (UPC++), Co-Array C++, 
OpenSHMEM, MPI-3 and Global Arrays.
These have the benefit of being integrated with existing languages, 
simplifying the learning curve for existing programmers.

The increasing availability of PGAS compilers and support software
opens up more opportunities than ever for researchers and developers
to test new strategies and port applications to more demanding
requirements. 

\section{Scope and Aims}

The scope of the PAW workshop is to provide a forum for exhibiting case studies of PGAS 
programming models in the context of real-world applications as a means of better 
understanding practical applications of PGAS technologies.  We encourage the submission 
of papers and talks detailing practical PGAS applications, including characterizations of 
scalability and performance, of expressiveness and programmability, as well as any downsides 
or areas for improvement in existing PGAS models. In addition to informing other application 
programmers about the potential that is available through PGAS programming, the workshop is 
designed to communicate these experiences to compiler vendors, library developers, and system 
architects in order to achieve broader support for PGAS programming across the community.

% The PAW workshop aims at bringing together members of the PGAS
% community;  we especially invite application specialists to
% share their experiences, and to exchange views with language and
% support researchers.

Topics include, but are not limited to:
\begin{itemize}
\item Novel application development using the PGAS model
\item Real-world examples demonstrating performance, compiler
  optimization, error checking, and reduced software complexity.
\item Performance evaluation of applications running under PGAS
\item Algorithmic models enabled by PGAS model
\item Compiler and runtime environments
\item Libraries using/supporting PGAS and applications
\item Benefits of hardware abstraction and data 
  locality on algorithm implementation.
\end{itemize}

\section{Submissions}
Submissions are solicited in two categories:
\begin{itemize}
  \item {\bf Full-length papers presenting novel research results:}
  \begin{itemize}
   \item[] Full-length papers will be published in the workshop proceedings in
cooperation with SIGHPC. Submitted papers must be original work that has not 
appeared in and is not under consideration for another conference or a journal. 
Papers shall not exceed eight (8) pages including text,
appendices, and figures. References are not included.
  \end{itemize}

  \item {\bf Extended abstracts summarizing published/preliminary results:} 
  \begin{itemize}
     \item[] Extended abstracts will be evaluated separately and are not intended 
     to prevent the work from being submitted to other forums for publication.  
     Extended abstracts shall not exceed four (4) pages. 
  \end{itemize}
\end{itemize}

Submissions shall be submitted through EasyChair; they must conform to ACM
Guidelines. Accepted full-length papers will be given longer presentation 
slots at the workshop than the abstract-only option. 


% \begin{itemize}
%         \item Full-length papers presenting novel research results.
% %  improving the state of the
% %   art or discussing the issues seen on existing extreme-scale systems,
% %   including  analysis and evaluation
%   \item Extended abstracts summarizing preliminary results or those
%           published previously.
% \end{itemize}
% Full-length papers will be published in the workshop proceedings in
% cooperation with SIGHPC and should constitute novel results from an
% academic publishing perspective.  Extended abstracts will be evaluated
% separately and are not intended to prevent the work from being
% submitted to other forums for publication.  Accepted papers will be
% given longer presentation slots at the workshop than the abstract-only
% option.

% Submissions shall be submitted through EasyChair; they must conform to ACM
% Guidelines.
% Regular papers shall not exceed eight (8) pages including all text,
% appendices, and figures.  Extended abstract papers shall not exceed
% four (4) pages. In the case of extended abstract, shorter submissions
% are acceptable for previously published work.

% BLC: I had been imagining 2 pages for extended abstracts, not that
% I'm opposed to 4.  Maybe we could say ``but shorter submissions are
% acceptable as well'' for the extended abstract case?  What I'm
% thinking of is that if you've done some great, previously published,
% work in PGAS and we want you to present it at PAW, we probably want
% to give you a low bar to show up and talk about it?


\subsubsection*{WORKSHOP CHAIR}
Karla Morris - Sandia National Laboratory 

\subsubsection*{ORGANIZING COMMITTEE}
\begin{itemize}
\item Katherine A. Yelick - Lawrence Berkeley National Laboratory 
\item Yili Zheng - Lawrence Berkeley National Laboratory 
\item Salvatore Filippone - Cranfield University 
\item Bradford L. Chamberlain - Cray Inc.
\item Bill Long - Cray Inc. 
\end{itemize}

\subsubsection*{PROGRAM COMMITTEE CHAIR}
Bill Long - Cray Inc. 

\subsubsection*{PROGRAM COMMITTEE}
\begin{itemize}
  \item Gheorghe Almasi - IBM
  \item Bradford L. Chamberlain - Cray Inc.
  % \item ??Barbara Chapman - Stony Brook University \& Brookhaven National Laboratory 
  \item Daniel Chavarr\'{i}a - Pacific Northwest National Laboratory
  \item Bert de Jong - Lawrence Berkeley National Laboratory
  \item Salvatore Filippone - Cranfield University 
  % \item ??Tarek El-Ghazawi - George Washington University 
  \item David Grove - IBM
  \item Jeff Hammond - Intel 
  \item Oscar Hernandez - Oak Ridge National Laboratory
  \item Amir Kamil - University of Michigan 
  \item John Mellor-Crummey - Rice University
  \item Karla Morris - Sandia National Laboratory
  \item Nick Park - DoD
  \item Damian W. I. Rouson - Sourcery Institute
  \item Lauren Smith - DoD 
  \item Katherine A. Yelick - Lawrence Berkeley National Laboratory 
  \item Yili Zheng - Lawrence Berkeley National Laboratory 
\end{itemize}  

\subsubsection*{IMPORTANT DATES:}
\begin{itemize}
  \item Submission Deadline: July 31, 2016 
  \item Author Notification: September 1, 2016
  \item Camera Ready:        October 1, 2016
  \item Workshop Date:       November 13--18, 2016
\end{itemize}  

\end{document}



